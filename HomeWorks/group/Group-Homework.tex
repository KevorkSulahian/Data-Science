\documentclass[]{article}
\usepackage{lmodern}
\usepackage{amssymb,amsmath}
\usepackage{ifxetex,ifluatex}
\usepackage{fixltx2e} % provides \textsubscript
\ifnum 0\ifxetex 1\fi\ifluatex 1\fi=0 % if pdftex
  \usepackage[T1]{fontenc}
  \usepackage[utf8]{inputenc}
\else % if luatex or xelatex
  \ifxetex
    \usepackage{mathspec}
  \else
    \usepackage{fontspec}
  \fi
  \defaultfontfeatures{Ligatures=TeX,Scale=MatchLowercase}
\fi
% use upquote if available, for straight quotes in verbatim environments
\IfFileExists{upquote.sty}{\usepackage{upquote}}{}
% use microtype if available
\IfFileExists{microtype.sty}{%
\usepackage{microtype}
\UseMicrotypeSet[protrusion]{basicmath} % disable protrusion for tt fonts
}{}
\usepackage[margin=1in]{geometry}
\usepackage{hyperref}
\PassOptionsToPackage{usenames,dvipsnames}{color} % color is loaded by hyperref
\hypersetup{unicode=true,
            pdftitle={Group Homework},
            pdfauthor={Tigran Sedrakyan},
            colorlinks=true,
            linkcolor=Maroon,
            citecolor=Blue,
            urlcolor=blue,
            breaklinks=true}
\urlstyle{same}  % don't use monospace font for urls
\usepackage{graphicx,grffile}
\makeatletter
\def\maxwidth{\ifdim\Gin@nat@width>\linewidth\linewidth\else\Gin@nat@width\fi}
\def\maxheight{\ifdim\Gin@nat@height>\textheight\textheight\else\Gin@nat@height\fi}
\makeatother
% Scale images if necessary, so that they will not overflow the page
% margins by default, and it is still possible to overwrite the defaults
% using explicit options in \includegraphics[width, height, ...]{}
\setkeys{Gin}{width=\maxwidth,height=\maxheight,keepaspectratio}
\IfFileExists{parskip.sty}{%
\usepackage{parskip}
}{% else
\setlength{\parindent}{0pt}
\setlength{\parskip}{6pt plus 2pt minus 1pt}
}
\setlength{\emergencystretch}{3em}  % prevent overfull lines
\providecommand{\tightlist}{%
  \setlength{\itemsep}{0pt}\setlength{\parskip}{0pt}}
\setcounter{secnumdepth}{0}
% Redefines (sub)paragraphs to behave more like sections
\ifx\paragraph\undefined\else
\let\oldparagraph\paragraph
\renewcommand{\paragraph}[1]{\oldparagraph{#1}\mbox{}}
\fi
\ifx\subparagraph\undefined\else
\let\oldsubparagraph\subparagraph
\renewcommand{\subparagraph}[1]{\oldsubparagraph{#1}\mbox{}}
\fi

%%% Use protect on footnotes to avoid problems with footnotes in titles
\let\rmarkdownfootnote\footnote%
\def\footnote{\protect\rmarkdownfootnote}

%%% Change title format to be more compact
\usepackage{titling}

% Create subtitle command for use in maketitle
\newcommand{\subtitle}[1]{
  \posttitle{
    \begin{center}\large#1\end{center}
    }
}

\setlength{\droptitle}{-2em}

  \title{Group Homework}
    \pretitle{\vspace{\droptitle}\centering\huge}
  \posttitle{\par}
    \author{Tigran Sedrakyan}
    \preauthor{\centering\large\emph}
  \postauthor{\par}
      \predate{\centering\large\emph}
  \postdate{\par}
    \date{July 21, 2018}


\begin{document}
\maketitle

\section{Quantitative Analysis}\label{quantitative-analysis}

\subsection{Getting the data}\label{getting-the-data}

For getting the quantitative data, we chose API requests to
\href{https://exchangeratesapi.io/}{Exchange Rates API}. This API
provides exchange rates of currencies for any day since 1999. We made
several types of date requests, but for any of them the base currency
was chosen the USD.

\subsection{Analysing the data}\label{analysing-the-data}

\includegraphics{Group-Homework_files/figure-latex/unnamed-chunk-1-1.pdf}

In the bar plot above we can see the latest exchange rates based on US
dollar. The x-axis shows the symbols for currencies and the rates are
depicted on the y-axis. According do the plot, we can say that after IDR
and KRW, HUF and JPY have the highest rates. Moreover, GBP has the
lowest rate. We can represent the same chart in a bit more elegant form.

\includegraphics{Group-Homework_files/figure-latex/unnamed-chunk-2-1.pdf}

However those chart stand for the most recent exchange rates. Let's see,
however, how the rate for EUR has changed, for the last 4 years.

\includegraphics{Group-Homework_files/figure-latex/unnamed-chunk-3-1.pdf}

This plot shows a general idea about the rates of EURO based USD from
2014 to 2018. For the plot, only a single day has been chosen each year
for a genral idea about the behavior of the rates thorugh years. We
enounter the hight rate in 2017 and the lowest 2014. We see a constant
increase with the exception of last year, when rate for Euro has
dropped.

\section{Qualitative Analysis}\label{qualitative-analysis}

\subsection{Getting the data}\label{getting-the-data-1}

For qualitative analysis part of our group homework we chose to scrape
data from Steam game store and service. For this purpose, we created a
dataframe with 4 columns: Game Id, Game Title, Vote and Review. The
first one shows the id of the game, the second one corresponds to the
title, third one is ``Recommended'' or ``Not recommended'', converted to
a number (1 and -1, correspondingly). The fourth one is the review left
by the user. We chose 5 games, and for each game we've scraped the most
recent 1010 reviews, with corresponding votes. It's worth noting that we
chose the 5 most played games at the moment, according to
\href{http://steamdb.info}{SteamDB} in the corresponding order of daily
players:

\begin{enumerate}
\def\labelenumi{\arabic{enumi}.}
\tightlist
\item
  PLAYERUNKNOWN'S BATTLEGROUNDS
\item
  Dota 2
\item
  Counter-Strike: Global Offensive
\item
  Warframe
\item
  Tom Clancy's Rainbow Six Siege.
\end{enumerate}

The game ids were extracted manually and kept in a vector over which we
later iterated. The rest of the data, including titles were scraped
using different techniques. The data was then saved into csv file, for
easier future usage.

Of particular interest is the column Review, which we had to clean of
the needless information, which could later interfered in the analysis
part. With every review came the first phrase, which contained the
`Posted:' + the date of posting of the review. Because we're doing pure
text analysis we thought that it'd be better to get rid of this text
This was achieved using regex patterns. Whitespaces, with the exception
of space, and non-ASCII characters were also removed, thus leaving pure
reviews, if such were present. All of the above can be found in file
which can be found in the scrape.R file attached.

In the end we got a dataframe that looks like this:

\begin{verbatim}
## Error in file(file, "rt"): cannot open the connection
\end{verbatim}

\begin{verbatim}
## Error in head(games): object 'games' not found
\end{verbatim}

\subsection{Analysing the data}\label{analysing-the-data-1}

First of all, let's see how review character length is distributed.

\begin{verbatim}
## Error in nchar(games$Review): object 'games' not found
\end{verbatim}

\includegraphics{Group-Homework_files/figure-latex/unnamed-chunk-5-1.pdf}

The picture is not surprising. Only very few users spend their time on
writing reviews with over 1000 charecter length. Mostly people manage to
fit their gratitude or anger in less symbols, that's why we have more
reviews with few symbols and less reviews with a lot of symbols. But how
does this plot look for each game individually?

\begin{verbatim}
## Error in eval(lhs, parent, parent): object 'games' not found
\end{verbatim}

As we see review length distribution for each game isn't much different
from that of all game altogether. Although we notice here that Counter
Strike and Dota 2, in contranst to the other 3 games, are concentrated
to the left, meaning that people spend the least time and charecters for
writing reviews for these games. We also notice that little spike around
3000 charecters for Warframe. Let's see which words are the most popular
in reviews of this game.

\begin{verbatim}
## Error in unique(games$Game.Title): object 'games' not found
\end{verbatim}

\begin{verbatim}
## Error in unique(game_titles): object 'game_titles' not found
\end{verbatim}

\begin{verbatim}
## Error in -freq: invalid argument to unary operator
\end{verbatim}

Its title `Warframe' is predominant here. We also notice terms like
play, free and good, indicating generally favourable reviews. But this
wordcloud doesn't tell us much about the game compared to the other 4
games in out list. Let's see how games compare in a comparison
wordcloud.

\begin{verbatim}
## Error in eval(expr, envir, enclos): object 'game_titles' not found
\end{verbatim}

\begin{verbatim}
## Error in term.matrix[, i]: subscript out of bounds
\end{verbatim}

The word `cheater' is instantly noticeable, as it's the most popular
one. As we see from the colors, it's related to game Playerunknown's
Battlegrounds (PUBG). We also notice words `hacker', `cheat' and `lag'
for this game, meaning that users probablt didn't like it that much.
Game names - Dota, Warframe, CSGO, PUBG and Siege are quite prominent
here, too. We also notice game developer company names, like Valve and
Ubisoft. However, we have quite a lot of unuseful words, like `game' and
`play', and several game titles. We can scale down importance of those
words and plot the wordcloud again:

\begin{verbatim}
## Error in eval(expr, envir, enclos): object 'game_titles' not found
\end{verbatim}

\begin{verbatim}
## Error in term.matrix[, i]: subscript out of bounds
\end{verbatim}

The `warfram' in the center indicates that most of the reviews in
general contained this word the most. We still have some company and
game names here. But notably for PUBG, we still have word `cheater'. We
also have another new noticeable word - `regionblockchina', indicating
the recent block of the game, by Chinese government. Judging by these
two words, the game's ratings, calculated by dividing number of positive
votes, by the number of total votes, as well as review polarity cannot
be high. Let's check it and compare with other games:

\begin{verbatim}
## Error in polarity(text.var = games$Review, grouping.var = games$Game.Title): object 'games' not found
\end{verbatim}

\begin{verbatim}
## Error in eval(lhs, parent, parent): object 'games' not found
\end{verbatim}

\begin{verbatim}
## Error in merge(rating, select(pol_score, c("ave.polarity", "Game.Title")), : object 'rating' not found
\end{verbatim}

\begin{verbatim}
## Error in eval(lhs, parent, parent): object 'polarity_rating_df' not found
\end{verbatim}

\begin{verbatim}
## Error in eval(lhs, parent, parent): object 'polarity_rating_df' not found
\end{verbatim}

\begin{verbatim}
## Error in arrangeGrob(...): object 'p1' not found
\end{verbatim}

As expected, PUBG has the lowest polarity, which drops below zero,
indicating that the reviews were mostly negative. In contrary to PUBG,
is already familiar Warframe, which has pretty respectable polarity
score of almost 0.3. We see approximately the same picture for rating
barplot. PUBG has the lowest rating of about 12-13 out of 100. Warframe
is again on the opposite side of the chart with rating of more than 90.
But Polarity and Ratings from those two charts seem correlated, don't
they?

\begin{verbatim}
## Error in ggplot(polarity_rating_df, aes(x = Rating, y = ave.polarity, : object 'polarity_rating_df' not found
\end{verbatim}

They are indeed, positively correlated as we see. Increase in one of
them implies the increase of the other and vice versa.

Looking at this graphs one might think that PUBG definitely cannot sell
good. However, one unexplainable phonomenon to humanity is how this game
manages to stay the
\href{https://store.steampowered.com/search/?filter=topsellers}{\#1 top
selling game in the world} for over a year, with over a million active
daily players. One of the possible factors is that it's one of the first
relatively big games in its
\href{https://en.wikipedia.org/wiki/Battle_royale_game}{battle royale}
genre.

We can make a conlusion that people pay more attention to the top
selling charts, rather than reviews left by users just like them. This
implies that if the game sells good doesn't necessarily mean that it's
the best one.


\end{document}
