\documentclass[]{article}
\usepackage{lmodern}
\usepackage{amssymb,amsmath}
\usepackage{ifxetex,ifluatex}
\usepackage{fixltx2e} % provides \textsubscript
\ifnum 0\ifxetex 1\fi\ifluatex 1\fi=0 % if pdftex
  \usepackage[T1]{fontenc}
  \usepackage[utf8]{inputenc}
\else % if luatex or xelatex
  \ifxetex
    \usepackage{mathspec}
  \else
    \usepackage{fontspec}
  \fi
  \defaultfontfeatures{Ligatures=TeX,Scale=MatchLowercase}
\fi
% use upquote if available, for straight quotes in verbatim environments
\IfFileExists{upquote.sty}{\usepackage{upquote}}{}
% use microtype if available
\IfFileExists{microtype.sty}{%
\usepackage{microtype}
\UseMicrotypeSet[protrusion]{basicmath} % disable protrusion for tt fonts
}{}
\usepackage[margin=1in]{geometry}
\usepackage{hyperref}
\hypersetup{unicode=true,
            pdftitle={HW\#2},
            pdfauthor={Karen Mkhitaryan},
            pdfborder={0 0 0},
            breaklinks=true}
\urlstyle{same}  % don't use monospace font for urls
\usepackage{graphicx,grffile}
\makeatletter
\def\maxwidth{\ifdim\Gin@nat@width>\linewidth\linewidth\else\Gin@nat@width\fi}
\def\maxheight{\ifdim\Gin@nat@height>\textheight\textheight\else\Gin@nat@height\fi}
\makeatother
% Scale images if necessary, so that they will not overflow the page
% margins by default, and it is still possible to overwrite the defaults
% using explicit options in \includegraphics[width, height, ...]{}
\setkeys{Gin}{width=\maxwidth,height=\maxheight,keepaspectratio}
\IfFileExists{parskip.sty}{%
\usepackage{parskip}
}{% else
\setlength{\parindent}{0pt}
\setlength{\parskip}{6pt plus 2pt minus 1pt}
}
\setlength{\emergencystretch}{3em}  % prevent overfull lines
\providecommand{\tightlist}{%
  \setlength{\itemsep}{0pt}\setlength{\parskip}{0pt}}
\setcounter{secnumdepth}{0}
% Redefines (sub)paragraphs to behave more like sections
\ifx\paragraph\undefined\else
\let\oldparagraph\paragraph
\renewcommand{\paragraph}[1]{\oldparagraph{#1}\mbox{}}
\fi
\ifx\subparagraph\undefined\else
\let\oldsubparagraph\subparagraph
\renewcommand{\subparagraph}[1]{\oldsubparagraph{#1}\mbox{}}
\fi

%%% Use protect on footnotes to avoid problems with footnotes in titles
\let\rmarkdownfootnote\footnote%
\def\footnote{\protect\rmarkdownfootnote}

%%% Change title format to be more compact
\usepackage{titling}

% Create subtitle command for use in maketitle
\newcommand{\subtitle}[1]{
  \posttitle{
    \begin{center}\large#1\end{center}
    }
}

\setlength{\droptitle}{-2em}

  \title{HW\#2}
    \pretitle{\vspace{\droptitle}\centering\huge}
  \posttitle{\par}
    \author{Karen Mkhitaryan}
    \preauthor{\centering\large\emph}
  \postauthor{\par}
      \predate{\centering\large\emph}
  \postdate{\par}
    \date{June 29, 2018}


\begin{document}
\maketitle

In this homework you will work on video games dataset containing
information about popular video games, their sales in North America,
Europe, Japan and globally in the world. In the dataset ratings by
critics and users are presented and ratings of the games.

\subsection{Solve the problems and submit the .Rmd
file.}\label{solve-the-problems-and-submit-the-.rmd-file.}

WARNINGS!!! (If not done you will lose points.) 1) Make sure to put
titles on the plots and texts on axes. 2) If the plot is not
interpretable, zoom on ``x'' or ``y'' axes to make the graph more
interpretable (P4,P5, P7 and P8).
--------------------------------------------------------

\section{P1)}\label{p1}

Import the dataframe in R and with the use of dplyr subset it using the
following information.

-remove columns Publisher, JP\_Sales (Sales in Japan), Critic\_Count,
User\_Count and Developer. (1p) -Multiply the numbers in NA\_Sales,
EU\_Sales and GP\_Sales by 1 million as they are given in millions of
sales. (1p) -include only those for which NA\_Sales\textgreater{}=20000,
EU\_Sales\textgreater{}=20000 and Ranking is among
Everyone(``E''),Mature(``M''), Teen(``T''), Everyone 10+(``E10+'') and
Adults Only (``AO''). (1p)

\begin{center}\rule{0.5\linewidth}{\linethickness}\end{center}

\section{P2)}\label{p2}

Use data cleaning tools to clean the data.

\begin{enumerate}
\def\labelenumi{(\alph{enumi})}
\tightlist
\item
  Look at the columns which are either numeric or integer. Make sure
  they contain only numbers or NA's (nothing else). (1p)
\item
  Critic scores can be from 0 to 100 and users scores from 0 to 10. If
  there are values not from these intervals clean that observations
  using ifelse statement. (2p)
\item
  Look at the Genres: check if all categories are unique and if not,
  clean them so that there are no duplicate names. (2p)
\end{enumerate}

\begin{center}\rule{0.5\linewidth}{\linethickness}\end{center}

\section{P3)}\label{p3}

Create a scatterplot displaying how User scores and Critics score are
interconnected -make the point shape triangle, color red and
transperancy 20\%. Explain what you see in the graph. (1p)

\begin{center}\rule{0.5\linewidth}{\linethickness}\end{center}

\section{P4)}\label{p4}

Construct a graph showing how the global sales of the game is dependent
of a score given by the user and explain what you see in the graph.
(Hint! ?options to display values without ``e'' short notation) (1p)

\begin{center}\rule{0.5\linewidth}{\linethickness}\end{center}

\section{P5)}\label{p5}

Make previous plot more appealing using the following. (1p) -x axis name
-- ``Score given by the User'' color red, bold size=15 -y axis name --
``Global Sales of the game'' color red, bold size=15 -points (shape -
square, color-red, size- 1.5) -title of the plot -- ``User Score versus
Global Sales'' - Make panel background color \#09f2d5 - axis texts bold
black

\begin{center}\rule{0.5\linewidth}{\linethickness}\end{center}

\section{P6)}\label{p6}

Create a histogram to find the distribution of the games by Genre. What
are the top 3 Genres. Rotate Genre names on ``x'' axis to avoid
overlapping text (Hint! ?element\_text, ?theme) (2p)

\begin{center}\rule{0.5\linewidth}{\linethickness}\end{center}

\section{P7)}\label{p7}

Define the Rating as Factor and use faceting to plot the User score of
the game versus the North America Sales for different Ratings. Make
comment about the results.(2p)

\begin{center}\rule{0.5\linewidth}{\linethickness}\end{center}

\section{P8)}\label{p8}

Create a boxplot where ``x axis'' represents the Genre and ``y axis''
the Global Sales of the video game for a particular Genre. Make the text
on ``x'' axis vertical (Hint! ?theme, ?element\_text). Make some
comments.(2p)

\begin{center}\rule{0.5\linewidth}{\linethickness}\end{center}

\section{P9)}\label{p9}

Zoom the previous plot (Numbers on ``y'' axis (0,2million)) to clearly
see the boxplots for each Genre and make comments. (1p)

\begin{center}\rule{0.5\linewidth}{\linethickness}\end{center}

\section{P10)}\label{p10}

Create a barplot using dyplr functionalities and faceting to show the
total Global Sales for each year for each Rating. (2p)

\begin{center}\rule{0.5\linewidth}{\linethickness}\end{center}

\section{P11)}\label{p11}

Use the pipe operator and functions from dplyr package and show the
number of video games in each genre in descending order. (2p)

\begin{center}\rule{0.5\linewidth}{\linethickness}\end{center}

\section{P12)}\label{p12}

Use dplyr to create a new variable (CU\_Score) in Video dataset which
for each video game will show the average of Critic score and 10* User
Score. (2p)

\begin{center}\rule{0.5\linewidth}{\linethickness}\end{center}

\section{P13)}\label{p13}

Use the pipe operator and functions from dplyr package to find the top 3
platforms and the number of video games developed for each of them. (2p)

\begin{center}\rule{0.5\linewidth}{\linethickness}\end{center}

\section{P14)}\label{p14}

We are interested in the number of video games developed for top
platforms for different years. Pick the top 3 platforms from previous
problem and make other platforms as ``Other'' using dplyr (Hint! ifelse
statement). Thereafter remove observations from dataframe which have NA
values (Hint! ?complete.cases).Now use faceting to draw the distribution
of games for each year for each platform. Make text on ``x'' axis
vertical and size=6. Make comments how the number of video games changed
for each platform for different years.(4p)

\begin{center}\rule{0.5\linewidth}{\linethickness}\end{center}


\end{document}
